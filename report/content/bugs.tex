\chapter{Known Bugs}

\paragraph{}
Being relatively new, NativeScript still exhibits some unexpected behaviours for which we had to find solutions. We now discuss some known bugs that have been bypassed or are still problematic.

\begin{description}[style=nextline]
\item[Compressed Files (Android)] Some NPM libraries are shipped with compressed files that usually end with \url{.gz} and are also included in their non-compressed form\footnote{For example \url{file.min.js} and \url{file.min.js.gz}}. Android interprets these as a same and single file, this means that at compilation time Gradle will throw a \url{duplicate} \url{resources} error. To bypass this bug, we automatically delete the compressed versions\footnote{These files are not needed at runtime.} before they get compiled. The corresponding commands are located in the \url{prepare-hook.sh} bash script.

\item[The BroRand Library\footnote{\url{https://www.npmjs.com/package/brorand}}] BroRand is a JS library used to generate random numbers. CPMAC makes indirectly use of BroRand through the EC library called elliptic\footnote{\url{https://www.npmjs.com/package/elliptic}}. BroRand is not fully supported on the NativeScript framework and throws an error at execution time. This bug has been bypassed by replacing the problematic lines. The corresponding commands are found in the \url{prepare-hook.sh} bash script and the modified code in the \url{brorand-fix/} folder.

\item[WebSocket Bug (iOS)] The websocket library for NativeScript does not work correctly when run on iOS. Currently no way has been found to correct this bug in any way. The used library is called \url{nativescript-websockets}\footnote{\url{https://www.npmjs.com/package/nativescript-websockets}}. Basically, this library wraps native websocket libraries for Android and iOS in JS objects. It seems that the wrapper implemented by \url{nativescript-websockets} is working properly, but that either the underlying native Objective-C library, which is a slightly modified version of PocketSocket\footnote{\url{https://github.com/NathanaelA/PocketSocket}}, or iOS itself is causing troubles. It seems that either PocketSocket or iOS resends previous messages by concatenating them with new data. Below are two sent messages and the corresponding received messages by a conode, they are displayed as bytes and in their base64 format to be easily readable. We verified that the sent messages are properly passed down to the native library by the JS wrapper. The first message is an empty PIN request, thus only contains the public key of the organizer and is correctly received by the conode. The second message, which now also contains the PIN, is not received correctly. The received message is the first message concatenated by six zeros and then the beginning of the second message.
\begin{description}[style=nextline]
\item[Sent]
\begin{lstlisting}
EiDLgkwNauV7pExX7QEpT3Zdu7z4nxWRnmRXdK9KvZnEfA==
18 32 203 130 76 13 106 229 123 164 76 87 237 1
41 79 118 93 187 188 248 159 21 145 158 100 87
116 175 74 189 153 196 124
\end{lstlisting}

\item[Received]
\begin{lstlisting}
EiDLgkwNauV7pExX7QEpT3Zdu7z4nxWRnmRXdK9KvZnEfA==
18 32 203 130 76 13 106 229 123 164 76 87 237 1
41 79 118 93 187 188 248 159 21 145 158 100 87
116 175 74 189 153 196 124
\end{lstlisting}
\end{description}
\begin{description}[style=nextline]
\item[Sent]
\begin{lstlisting}
CgY3NTIzMTMSIMuCTA1q5XukTFftASlPdl27vPifFZGeZFd0r0q9mcR8
10 6 55 53 50 51 49 51 18 32 203 130 76 13 106
229 123 164 76 87 237 1 41 79 118 93 187 188
248 159 21 145 158 100 87 116 175 74 189 153
196 124
\end{lstlisting}

\item[Received]
\begin{lstlisting}
EiDLgkwNauV7pExX7QEpT3Zdu7z4nxWRnmRXdK9KvZnEfAAAAAAAAAoG
18 32 203 130 76 13 106 229 123 164 76 87 237 1
41 79 118 93 187 188 248 159 21 145 158 100 87
116 175 74 189 153 196 124 0 0 0 0 0 0 10 6
\end{lstlisting}
\end{description}
As already stated before, no way has been found to bypass or fix this bug. Here is a non-exhaustive list of what has already been tried but didn't succeed:
\begin{itemize}
\item Find another NativeScript library to replace \url{nativescript-websockets}
\item Modify \url{nativescript-websockets} to try to reset the message buffer and others
\item Natively implement websockets by using the same PocketSocket version as \url{nativescript-websockets}
\item Natively implement websockets by using a different native library called SwiftWebSocket\footnote{\url{https://github.com/tidwall/SwiftWebSocket}}
\end{itemize}
\end{description}
