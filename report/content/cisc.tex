\chapter{CISC}

\paragraph{}
As it was shown earlier, right now the Cisc is only available on computers using the CLI developped by the DeDiS lab. The point of the Cisc is to have multiple devices connected to the skipchain for security reasons. Indeed, using the Cisc allow the user to spread its data, without giving the opportunity to an attacker to modifiy it without having access to a given threshold of device. That is why it was necessary to develop an application to allow user to manage their Cisc data on smartphones. This section will present the work done in the Cisc drawer of the app, the results that were achieved and some of the possible future work to improve the application.

\section{Implementation and Evaluation}

\paragraph{}
The implemetation of the Cisc drawer is done in a single JavaScript class Cisc.js. This object allows a user to connect to an existing skipchain that was created using the CLI. Once the device is connected to a skipchain it will access a first set of functionalities:
\begin{itemize}
  \item Access to the list of registered devices for the skipchain
  \item Possibility to browse the SSH-keys
  \item Possibility to browse the webpages stored on the skipchain
  \item Possibility to browse the key-value pairs
\end{itemize}

\paragraph{}
The device will have the possibility to submit a request to become a part of the registered devices. To do so it will propose a new block to the skipchain containing the same data as the one that are already on the chain, completed by a new entry in the device array. This entrt being its name and public key. If this block gets accepted, the device becomes registered and receive the right to vote for the proposed data.
The first thing to do when a user want to use the Cisc is to define a name in the parameters. Once this is done he will have the possibility to connect to a skipchain by using the button displayed in the drawer. When the button is tapped, the app turns into a QR code scanner. The QR scanner is expecting a QR formated as the one displayed using the following command in the terminal:
\begin{lstlisting}
cisc id qr
\end{lstlisting}

\paragraph{}
After scanning such code, the object will save it and try to access it to get the data from the identity skipchain. If this action is successful, the object will consider itself as connected to the skipchain. It will then try to see wether or not the device is part of the voting devices of the chain. If it is not it will propose a new block of data to add it.
Once the device is part of the voting devices, it will have the ability to vote for the proposed data.
To vote for an update, the user has to compute a hash of the proposed data. This hash is computed based on every data in the block, that is:
\begin{itemize}
  \item the threshold of the chain
  \item the devices name and public-keys alphabetically ordered
  \item the key-value pairs odered alphabetically
\end{itemize}
The vote then consists of a schnorr signature of this hash using the device private key, so that it is verifiable using the public key stored in the skipchain.

\section{Results}

\paragraph{}
As shown in the previous part, the app only use the basic functionalities of the Cisc implementation. The most important task was to allow a user to access the cisc on all his devices and this project is a good proof of concept.However, we can still imagine some work to complete the application.

\section{Future Work}

\paragraph{}
Some functionalities could improve the current implementation of the cisc. As example we can site the following:
\begin{itemize}
  \item Adding the possibility to connect to multiple skipchains: indeed, right now a user can only be connected to a single identity skipchain. It could be a good thing to add an option to save a group of available skipchains that the user already connected to.
  \item Creating identity skipchains: indeed, for the moment the user has to connect to an already existing skipchain that was created using the cisc CLI. A good improvement of the app could be to add the possibility to connect to conodes to create a skipchain. This would require an authentication system. Right now the Cothority accepts two types of authentifications:
  \begin{description}
    \item[Public] This authentification is done using a private/public keypair.
    \item[PoP token] This would be a good idea to add the possibility to create your own skipchain using directly your PoP Token that was generated from the other half of the application.
  \end{description}
\end{itemize}
