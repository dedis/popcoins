\chapter{Installation and Running of CPMAC}

\paragraph{}
We will now see how to install all required dependencies and how to compile, test and run the app. The following steps are:

\begin{enumerate}
\item Installation of Go Language

To be able to run the cothority framework you'll need the go compiler. Install it by following the official installation guide: \url{https://golang.org/doc/install} . The \url{GOPATH} environment variable has to be set by either following the official guide\footnote{\url{https://golang.org/doc/code.html#GOPATH}} or by running\footnote{Terminal restart needed.}:
\begin{lstlisting}
$ echo 'export PATH=$PATH:$(go env GOPATH)/bin'
>> ~/.bash_profile
\end{lstlisting}

\item Cothority Installation and Running of Conodes\footnote{\url{https://github.com/dedis/cothority}}

CPMAC is developped to run against the stable version 1.2 of Cothority. As stated in the \url{README.md} file on the GitHub page of the cothority framework, the version installed in \url{gopkg.in/dedis/cothority.v1} has to be used. The source code in this folder corresponds to the branch v1.2 located here: \url{https://github.com/dedis/cothority/tree/v1.2}. Clone the repository by running:
\begin{lstlisting}
$ go get -u github.com/dedis/cothority
\end{lstlisting}
Locate and enter the folder \url{gopkg.in/dedis/cothority.v1} in your \url{GOPATH}, you can then execute the following command to run three local conodes:
\begin{lstlisting}
$ ./conode/run_conode.sh local 3 5
\end{lstlisting}

\item Official NativeScript Tutorial\footnote{\url{https://docs.nativescript.org/start/quick-setup}}

It is recommended to follow the instructions on their official page, as it will always be up-to-date. But here are the main steps:
\begin{enumerate}
\item NodeJS Installation\footnote{\url{https://nodejs.org/en/}}

This can be done by downloading the installer on their home page. Always install a long term service (LTS) version as it is the supported version for NativeScript.

\item NativeScript CLI Installation\footnote{\url{https://www.npmjs.com/package/nativescript}}

This can be done by running the following command:
\begin{lstlisting}
$ npm install -g nativescript
\end{lstlisting}
If an EACCES error is returned at some point of the installation, re-run the last command with administrator rights. If EACCES errors are still returned, run the command again with administrator rights and the unsafe permissions parameter of NPM:
\begin{lstlisting}
$ sudo npm install -g --unsafe-perm nativescript
\end{lstlisting}

\item Android and iOS Requirements

Since it depends on your operating system (OS), follow the official tutorial\footnote{\url{https://docs.nativescript.org/start/quick-setup#step-3-install-ios-and-android-requirements}}. NativeScript provides scripts for Windows and macOS that will automatically setup most dependencies. It is still recommended to look at the advanced setups they provide to ensure that everything is correctly installed.

\item TNS Doctor

The last step is to check if all requirements are met, this can be done by running: \url{tns} \url{doctor}. If errors are returned, fix them before continuing.
\end{enumerate}

\item Editor

This step can be skipped if the goal is to only run the app but not to contribute to the project. In essence, any editor could be used. We recommend using Visual Studio Code\footnote{\url{https://code.visualstudio.com}} since it is the officially supported editor and provides an official plugin\footnote{\url{https://marketplace.visualstudio.com/items?itemName=Telerik.nativescript}} to integrate with NativeScript.

\item Install, Compile and Run CPMAC

Before going further make sure the conodes and an Android emulator or iOS simulator is set up and running. Clone the repository by executing:
\begin{lstlisting}
$ git clone https://github.com/dedis/
student_17_mobile.git
\end{lstlisting}
After entering the newly created folder \url{student_17_mobile}, test and run CPMAC by executing either one of the following commands:
\begin{lstlisting}
$ make clean-test-<platform>
$ make clean-run-<platform>
\end{lstlisting}
Where \url{<platform>} has to be replaced by either \url{android} or \url{ios}. This will install and compile all the needed libraries and test or run CPMAC. All subsequent tests and runs can be made by running:
\begin{lstlisting}
$ make test-<platform>
$ make run-<platform>
\end{lstlisting}
\end{enumerate}
