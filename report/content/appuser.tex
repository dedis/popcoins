\chapter{App User}
\paragraph{}
The user of the app is represented by a single class, the User.js object. This singleton handles all that is global to the app or the user and does not belong to CISC or PoP in particular, such as the roster displayed in the home drawer and the key pair managed in the user tab of the settings drawer. After the loading screen of the app, users land in the home drawer. Here, there are two main components: the roster of the user and a button to fetch the status of all of the user’s conodes. To add a new server, one can either enter the information manually or scan a QR code with either a JSON\footnote{\url{https://www.json.org}} or TOML\footnote{\url{https://github.com/toml-lang/toml}} description of the node. After being added to the roster, the status of all of the servers in the roster is fetched and can be re-fetched at any time with the corresponding button. To see the status of a conode, simply clicking on one of the conodes in the list opens a new page that contains all of the fetched stats of the conode. On top of this page is a button that permits the user to display a QR code to represent that particular conode. This permits convenient sharing of conodes, including when setting up a PoP party description.

\paragraph{}
The next functionalities of the user are located in the user tab in the settings drawer. Here, the user has the ability to generate a new key pair at any time, which is then displayed in this tab. Moreover, the user can display the key pair as QR code (with the private key removed). For example, this could be used for easy registration in a PoP party. The last button in the users’ settings completely resets any data related to the user, including the roster displayed in the home drawer and the key pair displayed in the settings.
